\Chapter{Bevezetés}

A digitális technológiák rohamos fejlődése új lehetőségeket teremtett az oktatás és a tudomány terén is. A felsőoktatási intézményekben egyre jobban elterjednek az informatikai megoldások, amelyek segítségével hatékonyabban tudják kezelni az adminisztratív feladatokat illetve támogatni a hallgatókat és az oktatókat a tanulás folyamatában. Az egyik ilyen kiemelkedő terület a szakdolgozatok és záróvizsgák menedzselése, amelyek jelentős mérföldkövei a hallgatók felsőoktatási pályafutásának.
\\
\\
Az ilyen alkalmazások lehetővé teszik a folyamatok hatékonyabb és átláthatóbb kezelését mind a hallgatók és az oktatók számára. Ennek segítségével könnyebb lehet az adminisztráció, az értékelés, valamint a kommunikáció az érintett felek között. A modern webalkalmazások rugalmas szerkezete lehetővé teszi az automatizált folyamatok bevezetését így segítve a hatékonyabb munkavégzést és az idő megtakarítását. Az ilyen megoldások a digitális oktatásban egyre nélkülözhetetlenebb szerepet töltenek be, hozzájárulva a hallgatók és az oktatók sikeres együttműködéséhez és az eredményes tanulási folyamathoz.
\\
\\
Ebben a szakdolgozatban egy olyan webalkalmazás koncepcióját és megvalósítását szeretném szemléltetni, amelynek célja a szakdolgozatok bírálati folyamatának és záróvizsgák menedzselésének támogatása. A rendszer tervezése és implementációja során figyelembe veszem az oktatók és a hallgatók igényeit, valamint a modern webfejlesztés korszerű trendjeit és technológiáit.
\\
\\
A továbbiakban áttekintem a jelenlegi helyzetet a szakdolgozatok és záróvizsgák adminisztrációjában, elemzek néhány már létező megoldást ezen a területen, majd részletesen bemutatom saját alkalmazásom architektúráját és funkcióit. Végül pedig kiértékelem a fejlesztés során szerzett tapasztalatokat és lehetőségeket a jövőbeni továbbfejlesztésre. A cél az, hogy egy olyan innovatív és hasznos eszközt nyújtsak, amely segíti az Alkalmazott Matematikai Intézeti Tanszéket a hatékonyabb és átláthatóbb adminisztrációban, valamint támogatja a hallgatókat a sikeres szakdolgozatok és záróvizsgák előkészítésében és lebonyolításában.
