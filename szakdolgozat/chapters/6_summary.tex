\Chapter{Összefoglalás}


A szakdolgozat és záróvizsga nyilvántartó rendszer egyre nélkülözhetetlenebb szerepet tölt be a modern oktatási intézményekben. Célom az volt, hogy az Alkalmazott Matematikai Intézeti Tanszék működését összehangoltabbá és hatékonyabbá tegyem, különös tekintettel a szakdolgozatok és záróvizsgák adminisztrációs feladataira.

Az általam kidolgozott rendszer számos előnyös funkciót kínál, amelyek nem csupán megkönnyítik a hallgatók és oktatók munkáját, hanem jelentősen javítják az intézet adminisztratív hatékonyságát is. Az adatbázisban történő szakdolgozatok és záróvizsgák tárolása például lehetővé teszi azok gyors és strukturált elérését.

A rendszer által biztosított felhasználói regisztráció és az egyedi felhasználói jogosultságok kezelése garantálja a rendszer biztonságos és személyre szabott használatát. Ezáltal minden felhasználó csak azokhoz az információkhoz és funkciókhoz fér hozzá, amelyek az ő szerepköréhez és feladataihoz kapcsolódnak.

A bírálati dokumentumok egységes kezelése jelentős előrelépést jelent az értékelési folyamatokban, segítve az oktatókat és vizsgabizottságot a hatékonyabb visszajelzések nyomon követésében és az eredmények rögzítésében. A szakdolgozatok letöltésének és törlésének lehetősége, valamint a záróvizsgák jegyzőkönyvének automatikus generálása és az osztályozás funkciója mind-mind hozzájárulnak az adminisztrációs folyamatok időhatékonyabbá tételéhez, növelve ezzel az intézmény átláthatóságát és hatékonyságát.

Bár a webalkalmazás jelenleg üzemkész állapotban van, még számos fejlesztési lehetőség rejlik benne. Az egyik ilyen lehetőség egy történetiséget leíró lap létrehozása, amely részletesen bemutatná a felhasználók által végzett műveleteket időre pontosan, például szakdolgozatok feltöltését vagy felhasználók felvitelét. Fontos lenne továbbá lehetőséget biztosítani a szakdolgozatok feltöltésénél a módosításra is, valamint bővíteni a profiloldalt olyan funkciókkal, mint a profilkép feltöltése vagy az egyéb adatok módosítása pl. e-mail cím.

A jövőbeli tervek között szerepel, hogy más tanszékek is használják az alkalmazást. Ennek részeként fontolóra fogom venni az alkalmazás továbbfejlesztését oly módon, hogy rugalmasan alkalmazkodjon más tanszékek specifikus igényeihez és adminisztrációs feladataihoz is. Ezen túlmenően, a felhasználói visszajelzések és igények alapján tervezek további funkciók és fejlesztések bevezetését, hogy az alkalmazás minél szélesebb körben hasznos és használható legyen a tanszékek számára. 

Egy Docker összerakása az alkalmazáshoz és egy OpenAPI leírás létrehozása a dokumentációk és felhasználói élmények optimalizálása érdekében kulcsfontosságú lépéseket jelent a rendszer fejlesztése során. A Docker konténerek segítségével egyszerűsíthetném az alkalmazás telepítését és környezetbe állítását, ami növeli a fejlesztési folyamat hatékonyságát és reprodukálhatóságát. Ezenkívül az OpenAPI leírás létrehozása lehetővé teszi számomra, hogy strukturált és könnyen értelmezhető dokumentációt készítsek az API-ról, ami segíthet az új fejlesztőknek a rendszeremmel való könnyebb és gyorsabb kapcsolatfelvételben. Ezáltal a fejlesztési folyamat átláthatósága és hatékonysága nő, miközben a felhasználói élmény javul, ami hozzájárul az alkalmazás sikerességéhez és elterjedéséhez.

Egy további fontos fejlesztési lehetőség a szakdolgozat és záróvizsga nyilvántartó rendszerben az automatizált backend tesztek összerakása. A JUnit tesztek segítségével lehetőségem lenne ellenőrizni a backend rész működését és stabilitását, ami hozzájárulhat a rendszer megbízhatóságának növeléséhez. 

Összességében a szakdolgozat és záróvizsga nyilvántartó rendszer rendkívül sokoldalú és potenciális fejlesztési irányokkal rendelkezik, amelyek elősegíthetik az oktatási intézmények hatékonyságát és versenyképességét a digitális korban.